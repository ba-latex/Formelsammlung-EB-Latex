\begin{minipage}{0.5\columnwidth}
    \renewcommand{\arraystretch}{1.05}
    \begin{table}[H]
        \begin{tabularx}{\columnwidth}{l l}
            Wert                & Formel \\
            $V_{ldb}$           & $=20\lg\frac{U_a}{U_e}$ (Verstärkung)\\
            $I_K$               & $=\frac{U_2-U_1}{Z_K}\quad I_1+I_K=0$\\
            $\underline{Z_1}$   & $=\frac{\underline{U_1}}{\underline{I_1}}=\frac{\underline{U_1}}{\underline{-I_K}}$\\
            $\underline{Z_1}$   & $=\frac{Z_K}{-V_0+1}$\\
            $U_1$               & $=\frac{U_2}{V_0}$ \\
            $z_2$               & $=\frac{z_K}{1+\frac{1}{-V_0}}$
        \end{tabularx}
    \end{table}
\end{minipage}
\begin{minipage}{0.5\columnwidth}
    \bild[1.0]{opv.png}{Platzh. für gut}{opv}
\end{minipage}
\subsection{Anwendungen mit frequenzunabhängiger Gegenkopplung}
    \subsubsection{Invertierender Verstärker/Umkehrverstärker}
        \begin{minipage}{0.6\columnwidth}
            \renewcommand{\arraystretch}{1.05}
            \begin{table}[H]
                \begin{tabularx}{\columnwidth}{l l}
                    Wert    & Formel \\
                    $r_e$   & $=R_1$ \\
                    $r_a'$  & $=r_a\cdot\frac{V}{V_0}$\\
                    $U_a$   & $=-\frac{R_2}{R_1}\cdot U_e$ \\
                    $V$     & $=-\frac{R_2}{R_1}$ \\
                \end{tabularx}
            \end{table}
            Einfluss der endlichen Verstärkung\\auf einen realen OPV:\\
            $V=-\frac{V_0\cdot R_K}{R_K+R_1\cdot V_0+R_1}$
        \end{minipage}
        \begin{minipage}{0.4\columnwidth}
            \bild[1.0]{opv-umkehrverst.png}{Platzh. für gut}{umkehrverstärker}
        \end{minipage}
    \subsubsection{Nichtinvertierter Verstärker/Elektrometerverstärker}
        \begin{minipage}{0.6\columnwidth}
            \renewcommand{\arraystretch}{1.1}
            \begin{table}[H]
                \begin{tabularx}{\columnwidth}{l l}
                    Wert  & Formel \\
                    $V$   & $=1+\frac{R_K}{R_1}=\frac{U_a}{U_e}$ \\
                    $U_a$ & $=(1+\frac{R_K}{R_1})\cdot U_e$ \\
                    $R_2$ & $=\longrightarrow R_K$\\
                    $r_e$ & $=r_{aL}$ \\
                    $r_a$ & $r_a\cdot\frac{v}{v_0}$\\
                \end{tabularx}
            \end{table}
        \end{minipage}
        \begin{minipage}{0.4\columnwidth}
            \bild[1.0]{opv-elmverst.png}{Platzh. für gut}{elmverst}
        \end{minipage}
    \subsubsection{Spannungsfolger}
        \begin{minipage}{0.6\columnwidth}
            \begin{table}[H]
                \begin{tabularx}{\columnwidth}{l l}
                    Wert  & Formel \\
                    $U_a$ & $=U_e$ \\
                    $V$   & $=1$ \\
                \end{tabularx}
            \end{table}
        \end{minipage}
        \begin{minipage}{0.4\columnwidth}
            \bild[1.0]{opv-spannungsfolger.png}{Platzh. für gut}{spannungsfolger}
        \end{minipage}
    \subsubsection{Addierverstärker}
        \begin{minipage}{0.6\columnwidth}
            \renewcommand{\arraystretch}{1.1}
            \begin{table}[H]
                \begin{tabularx}{\columnwidth}{l l}
                    Wert    & Formel \\
                    $-U_a$  & $=(I_1+I_2)\cdot R_0$\\
                            & $=\frac{R_0}{R_1}\cdot U_{e1}+\frac{R_0}{R_2}\cdot U_{e2}$ \\
                    $-U_a$  & $=(U_{e1}+U_{e2})\cdot\frac{R_0}{R_1}\leftarrow$ nur für $R_1=R_2$ !\\
                    $-I_K$  & $=I_1+I_2$\\
                    $V$     & $=1$ \\
                    $U_{e1}$& $=I_1\cdot R_1$ \\
                    $U_{e2}$& $=I_1\cdot R_2$ \\
                \end{tabularx}
            \end{table}
        \end{minipage}
        \begin{minipage}{0.4\columnwidth}
            \bild[1.0]{opv-addverst.png}{Platzh. für gut}{addierverstärker}
        \end{minipage}
    \subsubsection{Subtrahierverstärker}
        \begin{minipage}{0.6\columnwidth}
            \begin{table}[H]
                \begin{tabularx}{\columnwidth}{l l}
                    $-U_a$   & $=\frac{R_K}{R_1}\cdot U_{e1}-(1+\frac{R_K}{R_1})\cdot \frac{R_3}{R_2+R_3}\cdot U_{e2}$ \\
                    $U_a$    & $=U_1 + U_R$ \\
                    $-U_{a1}$& $=\frac{R_k}{R_1}\cdot U_{e1}$ \\
                    $U_{a2}$ & $=(1+\frac{R_k}{R_1})\cdot U_{e2}\underbrace{\cdot\frac{R_3}{R_2+R_3}}_\text{falls $R_3$ vorhanden}$ \\
                \end{tabularx}
            \end{table}
        \end{minipage}
        \begin{minipage}{0.4\columnwidth}
            \bild[1.0]{opv-subverst.png}{Platzh. für gut}{subtrahierverstärker}
        \end{minipage}
\subsection{OPV mit frequenzunabhängiger Gegenkompplung}
    \begin{minipage}{0.5\columnwidth}
        \subsubsection{Integrierverstärker}
        $-U_a=\frac{1}{RC}\cdot \int U_e(t) dt+U_0$\\ %korrektur laut OPV Stand 060321
        \bild[0.8]{opv-intverst.png}{Platzh. für gut}{integrierverstärker}
    \end{minipage}
    \begin{minipage}{0.5\columnwidth}
        \subsubsection{Differenzierer}
        $i_c=C\cdot\frac{du_c}{dt}\quad i_k=\frac{U_a}{R}=-i_c\quad -U_a=RC\cdot\frac{du_e}{dt}$\\
        \bild[0.8]{opv-differenzierer.png}{Platzh. für gut}{differenzierer}
    \end{minipage}
    \subsubsection{Komparator/Vergleicher}
        \begin{center}
            invertierter Komparator / nicht-invertierter Komparator
            \bild[1.0]{opv-komparator.png}{Platzh. für gut}{komparator}
        \end{center}
\subsection{Schmidtt-Trigger}
    \begin{minipage}[t][0.4\pdfpageheight][t]{0.5\columnwidth} %extra Argumente erzeugen 0.4xPapierhöhe hohe, top-aligned..te minipage
        \subsubsection{invertierter SWS}
        \renewcommand{\arraystretch}{1.1}
        \begin{table}[H]
            \begin{tabularx}{\columnwidth}{l l}
                Wert      & Formel \\
                $U_{ein}$ & $=\frac{R_2}{R_1+R_2}\cdot U_{a-}$ \\
                $U_{aus}$ & $=\frac{R_2}{R_1+R_2}\cdot U_{a+}$ \\
                $U_{Hys}$ & $U_{aus}-U_{ein}$
            \end{tabularx}
        \end{table}
        \bild[0.8]{schmidtt-inv.png}{Platzh. für gut}{sws-inv}
    \end{minipage}
    \begin{minipage}[t][0.4\pdfpageheight][t]{0.5\columnwidth}
        \subsubsection{nichtinvertierter SWS}
        \renewcommand{\arraystretch}{1.1}
        \begin{table}[H]
            \begin{tabularx}{\columnwidth}{l l}
                Wert      & Formel \\
                $U_{ein}$ & $=-\frac{R_1}{R_2}\cdot U_{a-}$ \\
                $U_{aus}$ & $=-\frac{R_1}{R_2}\cdot U_{a+}$ \\
            \end{tabularx}
        \end{table}
        \bild[0.8]{schmidtt-ninv.png}{Platzh. für gut}{sws-ninv}
    \end{minipage}
    \begin{minipage}[t][0.4\pdfpageheight][t]{0.5\columnwidth}
        \subsubsection{nichtinvertierter SWS mit Referenzspannungsquelle}
        \renewcommand{\arraystretch}{1.1}
        \begin{table}[H]
            \begin{tabularx}{\columnwidth}{l l}
                Wert      & Formel \\
                $U_{ein}$ & $=-\frac{R_1}{R_2}\cdot U_{a\min}+U_{ref}(1+\frac{R_1}{R_2})$\\
                $U_{aus}$ & $=-\frac{R_1}{R_2}\cdot U_{a\max}+U_{ref}(1+\frac{R_1}{R_2})$\\
            \end{tabularx}
        \end{table}
        \bild[0.8]{schmidtt-ninv-ref.png}{Platzh. für gut}{sws-ninv-ref}
    \end{minipage}
    \begin{minipage}[t][0.4\pdfpageheight][t]{0.5\columnwidth}
        \subsubsection{invertierter SWS mit Referenzspannungsquelle}
        \renewcommand{\arraystretch}{1.1}
        \begin{table}[H]
            \begin{tabularx}{\columnwidth}{l l}
                Wert      & Formel \\ %!beide Formeln in Original mit "-" vorn dran, laut Hannes nicht
                $U_{ein}$ & $=\frac{R_2}{R_1+R_2}\cdot U_{a\min}+U_{ref}(1-\frac{R_2}{R_1+R_2})$\\
                $U_{aus}$ & $=\frac{R_2}{R_1+R_2}\cdot U_{a\max}+U_{ref}(1-\frac{R_2}{R_1+R_2})$\\
            \end{tabularx}
        \end{table}
        \bild[0.8]{schmidtt-inv-ref.png}{Platzh. für gut}{sws-inv-ref}
    \end{minipage}